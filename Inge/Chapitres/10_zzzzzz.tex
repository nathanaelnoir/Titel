\chapter{Das junge Mädchen}
%\vspace{\baselineskip}
\lettrine{E}{twas} unruhig stand ich am Bahnsteig inmitten der um mich herum stehenden Menschen. Durch den Lautsprecher tönte die blecherne Stimme eines Mannes der den Ankommenden und Abreisenden sämtliche Verbindungen und Haltestellen bekannt gab. Es war einer dieser Tage nach den Osterferien, wo die Studenten und Schüler an ihre Studienorte zurückkehrten, mit schweren Koffern und ausgebeulten Rücksäcken beladen, die anderen -  Menschen, die zu irgendeinem Ort eilten oder müde die Heimreise antraten. \\\\
Ich öffnete die Abteilungstür. Die Nachmittagssonne zeichnete mit den ersten an diesem Tag  zaghaften Sonnenstrahlen Schimmer auf die verschmutzten Fensterscheiben und den überfüllten Gang. Befangen atmete ich den Geruch von Nervosität, Schweiß und Aufbruch. Kein Gruß, kein Nicken. Als der Zug abrollte, Taschen und Koffer verstaut waren, jeder einen Platz erhascht hatte, kehrte kurz die Stille des Abschieds ein. Ein Räuspern, ein Husten und die Handys wurden aus den Manteltaschen und Handtaschen gekramt. Wie immer ließ mich die Szenerie erschaudern. Der blasse Junge mir gegenüber starrte verbissen auf das Display, während gleichzeitig bei seinem Sitznachbar, undefinierbar eine Melodie spielte, mit klobigen Finger rollte er hastig durch die Nachrichten, äußerlich erinnerte er an einen derben Bauernjungen, er trug kurze helle Haare auf seinem Blumenkohlkopf. In der vorderen Reihe telefonierte ein Schwarzafrikaner lautstark, hörbar durch den ganzen Wagon, dabei federte er sein ausgestrecktes rechtes Bein mit den roten Sportschuhen, den Arm besitzergreifend auf der Sporttasche neben sich aufgelehnt. Zwischendurch zog er geräuschvoll durch die Nase. Er war vielleicht Mitte zwanzig.\\\\
Meinen Blick aus dem Abteilungsfenster gerichtet zog das Bild stadtauswärts an farblosen Häusern und dem Industriegebiet am Ende der Stadt vorbei. Der breite Fluss trug schwer am vollen Wasser aus den zufließenden Flüssen. \\\\
Noch am Morgen hing der Frühlingsnebel zwischen den Wäldern, unterhalb der schneebedeckten Berggipfel. Der Himmel glich einem zerfetzten grauen Schafwollteppich, wo hi und da ein Loch aufriss und der blaue Himmel heruntersah. Es hatte die Tage vorher noch heftig geregnet, auf den Höhen hielt der Winter mit Schneegestöber und Windböen hartnäckig seinen Vorrang. Jetzt leuchteten kurz die saftigen Wiesen und Hügel sattgrün, wie frisch gebadet und gebürstet. Weingärten lagen im Halbschlaf, blattlos auf deren Terrassen, sehnsüchtig warteten sie auf den Neubeginn. \\\\
Schweigend blickten die Leute auf das bläuliche Licht der Handys oder telefonierten - mit einer Ausnahme.\\\\
Auf der anderen Seite am Fenster saß ein junges Mädchen mit langen goldblonden glatten Haaren und Mittelscheitel. Wenn man ihr schmales Gesicht betrachtete, die Lippen waren zu einen geschwungenen Strich geschlossen, die schmale kleine Nase und die etwas vorstehenden Wangenknochen, die hohe Stirn und auch die langen geraden Augenbrauen wiesen auf einen wenig anpassungsfähigen Charakter hin. Insgesamt hatte das Mädchen ein regelmäßiges ovales Gesicht, einen zierlichen Körper für ihre überdurchschnittliche Größe, mit schmalen zarten Armen und wohlgeformten langen Beinen. Ihre Fingernägel waren gepflegt, das Make-up dezent. Man durfte sie als eine schöne Frau bezeichnen. \\\\
In ihren Händen hielt sie ein Buch. Wie eine Balletttänzerin mit anmutigen Bewegungen blätterte sie die gelesene Seite weiter, hielt inne, lächelte still und neigte den Kopf zur Seite. Man konnte erkennen auf alle Fälle arbeitete ihr Kopf schnell. Die gedruckten Sätze auf den Seiten ihres Buches wurden zu einer Einheit mit den lautlosen Geschichten der Landstriche die sich an beiden Seiten der Fenster abwechselten. Gerne würde ich deinen Gedanken lauschen. \\\\
Ausdrucklos schienen die Leute auf ihren Plätzen in der Geschäftigkeit der neuen Kommunikation. Saftlos wie Bäume und Sträucher im Spätwinter. \\\\
Menschen kommen und gehen.\\\\
Der Zug hält wiederholt an.\\\\
Das Kinn aufgerichtet, die Augen geradeausgerichtet, den Rücken gestrafft, mit einer schwarz-weiß karierte Jacke umhüllt, ihre große Umhängetasche und den rosa Rollkoffer in der Hand, stieg sie aus.\\\\
Der zu Anfang April noch kühle Wind auf dem belebten Bahnhof, auf welchem sie zügig dem Ausgang zusteuerte, ließ ihre Haare flattern und entblöße hin und wieder ihre feinen Ohren. \\\\

Es geht auch anders.
\\\\
Viel Glück junges Mädchen 
