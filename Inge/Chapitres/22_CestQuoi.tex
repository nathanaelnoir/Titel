\chapter{Das Holzgatter}
%\vspace{\baselineskip}
\lettrine{A}{us} der Winterreise. Ich muss oft an die Kuh denken an jenem Septembertag, die Schwarzweiße hinter dem Holzgatter, die den Kopf mir zuwendet und weiterkaut. Braune Glasmurmelaugen schauen durch gebürstete Wimpern, die raue Zunge schleckt an meinen Fingerspitzen. \\\\
Im späten Sommer wuchern unter den hängenden Wildrosenbüschen in Wildhaufen die Brennnesseln und ermatteten Schafgarben. Der schlagende Schweif verscheucht die lästigen Fliegen, trockener Kuhdung bröselt und fällt von den Flanken. Nahe den Steinmauern summt es, gedämpft von den Bienen und Hummeln die Blüten der gelben Goldruten umkreisen, murmelnd, wie in sich selbst versunken. Gegen Süden ist das Tal bereits düster, in der Ferne liegen überschattet Felder, Wiesen und Wälder. Der Spätnachmittag ist abgestumpft. \\\\
Drüben gemauert im idyllischen Dorf auf dem geschorenen Hügel steht glänzend der Kirchturm und glänzend dein Haus, mit den Geranien auf den Holzbalkonen, -  vielleicht -  und deiner Frau und dem Kind dem du zulächelst, - vielleicht -. Die abgeschwächte Sonne streichelt zärtlich über die nackten Bergwände. Kuhglockenklänge tänzeln lieblich über die abgedunkelten Wiesen.\\\\
Am Lindenbaum lehnt er und winkt mit der Hand.\\\\
Dann prallt es auf mich ein das Lied der Winterreise, ich kann es hören, nach und nach entsteht ein Bild. Wenn ich die Augen schließe erreicht mich ein dunkles Fließen, das Toben von harten Regentropfen, Sturmwinde und Kälte knallt auf meine Haut, Eiszapfen klammern sich in die Falten der Seele, Wellen erstummen unter den Steinen. Wie gerne hätte ich dir davon erzählt. Aber es war mit meiner Sprache nicht zu erklären. \\\\
Noch wie damals klingt es von dir in meinen Ohren, beschämend heute wie dort: \\\\
Hörst du nicht?\\
\begin{itemize}
	\item Die Oktavenverschiebung gibt dem Lied einen schaurigen Charakter
	\item Hier modulliert die Musik  - Durtonart
	\item Die elende Stimmung wird mit dem tief gesetzten Schlussakkord deutlich
	\item Ein durchgängiges Achtel
	\item das Klavier eine oktavversetzte Bewegung spielt
	\item Das Zittern des Blattes durch ein Tremolo ausgedeutet, das Fallen durch eine fallende Bewegung im Bass
\end{itemize}

Das junge Reh hat der Wald verschluckt, die Rosenknospe nie aufgeblüht, vertrocknet hängt ihr Köpfchen am Geäst.\\\\
Eine Minute stand ich noch da mit starren Augen und atme abgestorbene Luft ein.\\\\
Da schloss der Himmel. Ein Ahornblatt fällt auf den Boden.