\chapter{Die Kirchbank}
%\vspace{\baselineskip}
\lettrine{E}{r} war Kapuzinerpater, er war jung, er war schön, er war unser Religionslehrer in der Volkschule, alle waren in ihn verliebt, am meisten ich. \\\\
Pater Rupert.\\\\
Die Religionsstunden waren Geschichten erzählen von Gott, und das konnte Pater Rupert gut. \\\\
Jesus in seinen armseligen Lumpenkleider, wie er zu den Menschen spricht und Pater Rupert in seiner Kutte wie er zu uns Kindern spricht. \\\\
\textit{In der Pfarrkirche zum St. Michael in der Stadt des Mädchens, saßen die Jungs und die Mädels noch getrennt nach Geschlechtern in der linken und rechten Reihe der Kirche. Der Mittelgang trennte wie ein mächtiger reißender Strom das Männliche vom Weiblichen. Die Kindermesse und die neun Uhr Messe am Sonntag hielt die Kinder an, brav zum Gottesdienst zu erscheinen, bei Schnee und Regen, die oben am Berg und die unten in der Stadt. \\\\
Die Lichstrahlen brachen durch die bunten Glasfenster der Seitenfenster in den Hochaltar und zauberten eine sanfte Helligkeit in das Kircheninnere. Staubkörnern tänzelten durch den Lichtschein. Das kämpfende Bild des Luzifers mit dem heiligen Michael oberhalb des Tabernakels schien milde. Eine weiße sorgfältig bestickte Decke, getreu und sorgsam von der Mesnerin gebügelt umhüllte den schlichten Tisch des Herrn. Golden glitzerten Kerzenständer und Messgeschirr. Es roch nach Weihrauch, Andacht und Ehrfurcht.\\\\
In der Reihe hinter dem Mädchen, saßen die von der Stadt. Karin war acht Jahre und sie war hübsch, sie trug eine rote Schleife im Haar und rote Lackschuhe, obwohl es draußen Winter war. Man ahnte den Aufwand, den ihre Mutter betrieben hatte. Sie war in einen blauen Mantel eingehüllt und den Kopf wärmte eine weiße Plüschmütze, mit weißen Bommeln, die man unter dem Kinn binden konnte. \\\\
Das Mädchen steckte verstohlen ihre vom Schneematsch tropfende Schnürschuhe unter die Fußbank, knöpfte sich den bescheidenen Mantel zu, legte die derbe Wollmütze auf die Kirchbank und wippte verlegen mit den Beinen.\\\\
Sie liebte die Muttergottes am linken Flügel der Kirche. Das schmerzende Gesicht der Mutter, mit ihrem toten Sohn auf den Knien. \\\\
Das spöttische Tuscheln der Mädchen in der hinteren Reihe wird lauter, die hämischen Blicke brennen auf ihrem Rücken, treffen ihr Herz.\\\\
Da neigt sich ihr Blick zur Muttergottes und zum Pfarrer in der Kirche der dort steht in seiner reich verzierten Soutane. Lautlosen Schreien des Seelenklagens bekleidet den Raum, Kerzen zündeln , und der Chor singt ihr erhabenes Halleluja.}
\\\\
Das Mädchen, es ging dann nicht mehr zur Messe. Danke Mutti dafür.
\\\\
Morgen kaufe ich mir rote Schuhe.