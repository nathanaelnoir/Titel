\chapter{Die „Morzenmartha“}
%\vspace{\baselineskip}
\lettrine{S}{ie} schob ihr altes Fahrrad aus dem Holzschuppen, das verfilzte Haar wirr zu einem Knoten hochgesteckt. Hier oben auf der Erhebung und unten in der Stadt kannte sie jeder. \\\\
\textit{Auf dem Bauernhof hinter unserem Haus breiteten sich reife Wiesen aus, zwischen jungen Äpfel- und Kirschbäumen in freundlicher Nachbarschaft mit den alten Weinreben. Der Hof war respektabel groß, aber noch größer war unsere Furcht vor dem Bauern und dem knurrenden Hund. Der Bauer trug einen eindrucksvollen Schnauzbart mit hochgezwirbelten Enden und stolzierte in aufgeblähten Schritten durch das Feld. Er hatte eine Frau, Kinder und eine Schwester – die Martha. \\\\
Martha zog häufig bei unserem Haus vorbei, einmal fahrend auf ihrem Rad oder dickbeladen mit Säcken und Kübeln schiebend. Ihr trauriger Körper steckte in lumpigen Kleidern und löchrigen Stricksachen, das Schuhwerk an den Füßen war zerrissen. Der Anblick für uns Kleinen erschreckend, für manch Älteren eine beschämende Erinnerung an die Bilder der geschundenen Kreaturen im Krieg. \\\\
Sie gehörte zum Stadtbild, zum täglichen Leben. Man war an ihren Anblick gewöhnt, so wie man sich an das Straßenschild in der Ecke, oder den Hydranten vor dem Haus gewöhnt hatte. \\\\
Martha blieb nie stehen, sprach nie ein Wort. Sie war nie böse. Mit gesenkten Kopf ging sie an den Menschen vorbei. \\\\
Egal ob der Himmel im Herbst grau verschleiert, und an anderen Tagen schwer des kommenden Regens, Martha befuhr auf ihrem Fahrrad die gewohnten Wege. \\\\
Hatte man seine Haare nicht ordentlich frisiert, oder die Kleider schmuddelig, dann hieß es mahnend: „Du schaugsch aus wie die Morzenmartha, richt di her“.\\\\
Aber was wussten die lästernden Frauen an den Gartenzäunen und die spöttischen Kinder in den Gassen.\\\\
Etwas hatte sie aus der Bahn geworfen. Von einer großen Liebe die sie nicht leben durfte, erzählte man. Das Leben hatte es nicht mit ihr gut gemeint.}\\\\

Heute wünscht ich mir, mehr Menschen wären nicht so angepasst, würden auf Konventionen pfeifen, damit man sich wieder an die Marthas erinnert. \\\\
Aber vielleicht lächelt sie uns heute zu, dort wo sie jetzt ist und hält die Hand ihres Liebsten.