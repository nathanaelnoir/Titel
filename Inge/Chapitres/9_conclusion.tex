\chapter{Peter}

\lettrine{M}{an} trifft sich im Leben zweimal.
\textit{Sie hockte auf einer Holzkiste auf dem Balkon in einem kurzen blauen Hemdkleidchen und dunklen Stutzen, die blonden langen Haare waren zu zwei Zöpfen geflochten. Niemand war bei ihr, außer einem alten Puppenwagen und einer Ansammlung von sitzenden Püppchen. Inmitten der Gesichter stoch eine Puppe hervor, Peter. \\\\
Peter war groß, er war hässlich, mit weit ausgestreckten starren Beinen und Armen, und den gespenstisch unbeweglichen wasserblauen Augapfeln überragte er die Puppenschar. Anstatt der üblichen Puppenhaare, schoben sich Wellen aus Plastik wie Sanddünen über den großen Kopf. Selbst das gestrickte Jäckchen mit den gelben Entchen gab ihm nichts Anmutiges. Herrisch vergrub er alles um sich. Sein lächerliches Grinsen sog die Umgebung mit ins Teuflische. Er drückte das liebenswürdige der putzigen Köpfchen mit roten Schleifchen im Haar und den niedlichen Kleidchen weg. Gretchen, Lara, Lisa und Marie verschwanden neben seiner Gegenwart. Seine steifen Füße lugten durch die Bettdecke. Der rechte große Zeh hatte einen Riss. Die Aussicht auf die gemähten Wiesen hinter den Mauern und dem freundlichen Ploseberg wog das Dunkle an diesem Spätsommertag auf. \\\\
Sie mochte ihn nicht. Eigentlich.\\\\
Im anderen Spätsommer viele, viele Jahre später, da war sie schon eine erwachsene Frau, traf sie Peter. Der feine Herr im schmutzigweißen Leinenanzug, knapp von Statur, war zwischen den Bänken aufgestanden und sah mit fliegenden Augen an ihr vorbei und platzierte sich mitten im Saal. Seine schlohweißen Haare flogen in Wellen um das Gesicht. Sonnengegerbte Falten hatten sich wie die Furchen im Frühling auf den braunen Erdackern in die Haut am Hals, Wangen und Stirn gegraben. Die kleinen meerblauen Pupillen zuckten wie Stecknadelköpfe in einem elektrischgeladenen Gewitter. Beobachtend, unumschränkt ordnete er sein Blickfeld. Die Frauen entrückten in Unsicherheit neben seiner Präsenz. Später dann begegneten sie sich öfters, man sprach kurz miteinander und sie lauschte dann den Geschichten, die er erzählte, sie lauschte dem Erlebten, das er erzählte. Er erzählte immer von sich. Es spielte sein Stück, als Hauptdarsteller auf seiner Bühne mit den Händen fuchtelnd und einem ruhelosen Gemüt, ständig zum Sprung bereit auf der Flucht, ob von den Menschen oder sich, das wusste selbst er nicht. Manchmal zwischen den langen Zeiten des Wartens, saßen sie in einem Kaffee und tranken einen Cappuccino. Vom Fenster in jener Bar konnte sie es glänzen sehen, denn die Sonne war durch eine Nebelwand gedrungen und schien auf ihre Seele. Sie erriet noch nichts von den drückenden Wolken, die über ihr zusammenbrachen um dann erschöpft in die Knie zu gehen. Vielleicht entstand eine leichte Zuneigung und sie ahnte es nur, sein Lächeln sog die Umgebung mit ins Teuflische. 
Sie mochte ihn. Eigentlich.}\\\\

Irgendwann waren beide in ihrem Leben abhandengekommen, wie der aufsteigende Dampf auf heißen Pflastersteinen nach einem kraftvollen Sommerregen. Unmerklich.\\\\

Und es ist gut so.