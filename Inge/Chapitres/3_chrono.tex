\chapter{Alles in Ruhe}

\lettrine{N}{ur} noch ein Schritt über die speckige Steinplatte, poliert von den vielen Tritten der Frauen, Männer und Kinder im Bittgang, dabei schaue ich auf meine Schuhe, wie sie da stehen. Sie sind schwarz, schwarz wie der Geruch aller Kirchen. Angewurzelt an den dicken hohen Mauern, und Bodenplatten, die Hoffnungs- und Vergangenheitstrauer der lebenden und toten Seelen. Das Dunkle hat sich durchgefressen ist hineingekrochen, wie der Ruß in den Kaminen. Die schwere Eingangstür ist weit geöffnet. Hinter mir kitzelt ein Sonnenstrahl den Nacken, die Wärme malt Kreise über die Schulter, den Oberarm bis zu den Fingern der rechten Hand. Kühle Luft aus der Kirche strömt auf das Gesicht, ich atme tief ein und warte noch. Menschengemurmel rechts und links neben mir. Nur noch ein Schritt.\\\\
\textit{Die Raffrollos sind heruntergelassen. Zwischen dem Vorhang der Balkontür fällt das Licht der Straßenlaterne und den beleuchteten Fenstern der Nachbarn.  Aus dem Radio dröhnt Musik. Die Kinder inszenieren ein Darbietung. Das blonde Mädchen hat die langen Haare zu einem Rossschwanz hochgebunden. Sie steckt in einem Minirock  und einem lila Top, die Lippen sind rosa bemalt. Der Junge trägt einen übergroßen orangen Hut, der ihm immerzu ins Gesicht fällt, in der Hand baumelt ein Mikrophon. Die Show kann beginnen. \\\\
Die Luft ist wie aufgeladen von der Anspannung. Ich beobachte den Jungen mit dem breiten Lachen, die spitze Nase des Mädchens, die hinter dem Vorhang hervorlugt. Noch versteckt und abwartend für den Auftritt. Beim Anblick ihrer Fröhlichkeit, ihrer zarten Körper, der leichten Beweglichkeit durchströmt mich etwas Zärtliches. \\\\
Ins Sofa hineingedrückt, wie ein dicker Zwerg, schaue ich dem Schauspiel zu, den einstudierten Tänzen und höre die nachgesungenen Lieder von Brithney Spears, Avril Lavigne und Pussy Cat. \\\\
Die Schultern und Hüften hüpfen im Rhythmus, die Arme schwenken einmal gebeut dann gestreckt nach oben und unten. Die Haare wehen. Zwei, drei Schritte vor, zurück, herumgedreht. Nocheinmal. \\\\
Jetzt fühle ich mich hellgelb, hell, heller und  ein bißchen lila wie das Top. \\\\
Irgendwie habe ich es doch geschafft. Allein. Denke ich.\\\\
Später am Abend, das Wasser im Bad läuft noch, trotzdem kann ich sie hören, sie lachen und necken sich. Tropfen klatschen auf den Fliesenboden, etwas fällt um. Nachher, den Kopf auf die Kissen gepresst summen sie wiederholt die Melodien. Dann wird es still. Auf der Zimmerdecke bleiben sie kleben die Musiknoten, wie Glitzersterne in einer klaren Nacht. Heute darf ich das Fenster nicht öffnen, kommt mir in den Sinn. Leise schließe ich ihre Zimmertür. }\\\\
Nur noch ein Schritt. \\\\
Durch die Seitenfenster im Hochaltar strahlt ein gelbliches Licht. Mir gefällt es. Es tröstet mich. \\\\
Und ich kann sie deutlich sehen, die zwei jungen Lämmchen, die springen in Übermut über die nassgrünen Wiesen. Die Ohren baumeln im Wind. Birken biegen und beugen dazu melodienleicht ihre Äste und Stämme. Gleich weißen Wollknäueln verschwinden sie darauf. Freudig, Angstlos. \\\\
Den Wollfaden – den halt ich noch immer in meiner Hand. 
