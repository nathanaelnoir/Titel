\chapter{Ein Augenschlag}

\lettrine{E}{ine} Zeit lang stand sie schon unschlüssig am Fenster im dritten Stock. Die Straßenbeleuchtung warf spärliches Licht auf den feuchtnassen Asphalt. Der Sturm war vorüber. An der Kreuzung am Ende der Ausfahrt huschten die Scheinwerfer der Autos betriebsam aneinander vorbei - gegen Süden und Norden. Der Zebrasteifen schimmerte in mattem Gelb. Die Lichterkette der Bar auf der gegenüberliegenden Straße wiegten sich im müde werdenden Wind. In der Wohnung nebenan ging ein Licht an. Es war eine dunkle Nacht, bedrohliche schwere Wolken sperrten wie eine mächtige Mauer das Gute von dem Bösen. Der Mond schlief versteckt im Niemandsland, hatte nichts gutzumachen. \\\\
\textit{Sie öffnet das Fenster. Fährt sanft mit kreisenden Bewegungen über den Bauch. Das Kind in ihrem Leib bewegt sich nicht. Sie hat Angst. Man hat sie aus dem Garten Eden vertrieben. Die Äpfel verdorben, Dornen und Disteln bohren sich durch die Haut, der Engel geflohen. Die Uhr in der Küche stehengeblieben. Es ist Mitternacht. Ein Augenschlag, ein Fall. Nichts ist übriggeblieben.\\\\
Da regt sich wimmernd im elterlichen Zimmer der kleine Junge. Das ungeborene Kind boxt in ihrem Bauch. Aufgewacht.\\\\
Dann geht sie in das Zimmer, legt sich zu ihren Sohn, hält tröstenden die kleine Hand, spricht mit den Kindern und lächelt. }\\\\

Für meinen Sohn Elias\\
Für die stillen Schreie