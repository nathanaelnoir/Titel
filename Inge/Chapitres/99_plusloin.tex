\chapter[Aller plus loin]{Aller plus loin...}


Pour plus d'infos, textes et rendez-vous sur la lutte :

\begin{itemize}[leftmargin=*]
\item \url{https://zad.nadir.org}: le site du mouvement d'occupation. Vous y retrouverez de nombreux textes, communiqués, analyses. Sur la période des expulsions on peut citer:
\begin{itemize}[label=$\cdot$] 
\item Contre l’aéroport et son monde — échos de deux mois d’expulsions et de résistances sur la Zone À Défendre (brochure de janvier 2013)
\item Chronologie des actions directes en solidarité avec la zad (brochure de novembre 2012)
\item Et parce qu'on parle d'opérations policières à nouveau, vous pouvez consulter <<L'appel de la zad face aux menaces d'expulsion>>, publié en octobre 2016.
\end{itemize}
\item \url{https://www.acipa-ndl.fr}, site de la principale association citoyenne anti-aéroport, le meilleur endroit pour trouver les analyses juridiques et économiques.  
\item \url{https://naturalistesenlutte.wordpress.com}, pour les questions environnementales.
\item \url{https://constellations.boum.org}, le site d'un collectif d'écriture: \textit{la Mauvaise Troupe}. On y trouve une série d'entretiens avec des personnes en lutte sur la zad de Notre-Dame-des-Landes et dans la vallée de Susa contre le TGV Lyon-Turin (sous forme de brochures téléchargeables et imprimables), une sélection subjective de textes du mouvement anti-aéroport, sur la période des expulsions, des récits des combats contre l'avancée du projet ainsi que des outils pour penser l'avenir sans aéroport et questionner l'idée de commune.

Le collectif a également publié sur le sujet trois livres disponibles en librairie:
\begin{itemize}[label=$\cdot$] 
\item <<Défendre la ZAD>>, l'éclat, 2016
\item <<Contrées, histoires croisées de la ZAD de Notre-Dame-des-Landes et de la lutte No TAV>>, l'éclat, mai 2016
\item <<Saisons, nouvelles de la zad>>, l'éclat, septembre 2017
\end{itemize}
\item \url{https://whenthetrees.noblogs.org},  qui rassemble des films et un zine faits en 2012 autour des forêts de la zad avant et après les expulsions.
\item \url{http://paroledecampagnes.blogspot.com}, un blog tenu par un couple de paysan·nes vivant sur la zone et refusant de céder.
\item \url{http://www.isabellerimbert.fr/tous-camille}, une série de portraits photos sensibles d'opposant·e·s
\item Le film <<Le dernier continent>>, V.Lapize, 2015. Tourné de 2012 à 2014, il fait le portrait subjectif de la ZAD de Notre-Dame-des-Landes et de ses habitant·e·s.
\item Plus récemment, le film <<les pieds sur terre>> dresse un tableau sensible du petit hameau du Liminbout et de ses habitant·e·s. 
\end{itemize}

Il y a probablement un comité local près de chez vous à partir duquel s'organiser ou quelques personnes et énergies complices avec lesquelles le constituer.\\

\textbf{... venir plus près! }\\

Il y a aussi de multiples occasions de venir sur la zad, pour des chantiers, fêtes, ateliers, banquets, ou actions. N'hésitez pas à y passer, pour quelques jours ou plus longtemps !