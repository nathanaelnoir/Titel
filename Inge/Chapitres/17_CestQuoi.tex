\chapter{Die Puppe}
%\vspace{\baselineskip}
\lettrine{G}{egenüber} unserem Haus wohnte meine beste Freundin Patrizia in einem großen Haus mit großen Räumen und großen Holzbalkonen.\\\\
In den siebziger Jahren galt es als schick und modern einen Teppichboden im ganzen Raum verlegt zu haben, so auch bei meiner Freundin in einem mausigen Grau. Es kam heißes Wasser aus dem Hahn, ohne Feuer machen zu müssen. Im Keller gluckerte ein Heizanlage aus Öl, welche das Wasser durch Rohre in die Heizkörper pumpte. Durch alle Stöcke in alle Zimmer. Die Küche zierte eine neue Einbauküche, ein fortschrittlicher Herd, ein großer Kühlschrank und flotte elektrische Küchengeräte. Es war die Zeit des Wohlstands, nach dem Aufbau, dem Angekommen sein. Nur nicht für uns.\\\\
\textit{An manchen Tagen, spielten die zwei Mädchen im Dachboden der Freundin, dort gab es viele Dinge zu entdecken, man steckte die Köpfe durch die Dachluke, wühlte in alten Stehschränken und Truhen durch abgelegte, raffinierte Kleider aus den 50iger und 60iger Jahren der Mutter. Probierte sie an, drehte sich im Kreis und beschmierte sich mit rotem Lippenstift und blauer Augenschminke. Zog sich die Schuhe mit hohem Absatz über und stöckelten durch Staub und Spinnenweben. Modeschmuck aus falschen Perlen wurden umgehängt und funkelnde Ohrgehänge angesteckt. Zwei Mädchen im Fräuleinrausch.\\\\
In einer Ecke des Dachbodens lagen lieblos Spielsachen: armlose Teddybären, zerkratztes Puppengeschirr, verlorene Kartenspiele und eine Wackelpuppe: der  “Cicciobello“. Eine Errungenschaft der Neuzeit, nach starren Puppen mit stieren Augen und Gliedern, ein Wunderwerk mit einem Mund, den man ehrgeizig mit einem Fläschchen füttern konnte, sein weicher Körper wiegte sich geschmeidig in glücklichen Kinderarmen.\\\\
Cicciobellos beste Zeit war hinüber. Die Haare angezündelt, versengt - aus Zerstörungswut oder Langeweile, der Kopf in Schieflage, der Traurigkeit seines Daseins zollend, Arme und Beinchen beschmiert wartend auf die Liebe einer kleinen Puppenmutter. \\\\
Das Mädchen bat, die Puppe mitnehmen zu dürfen.\\\\
In der Schulaufführung zu Weihnachten stand das alljährliche Krippenspiel an. Die Kinder huschten nervös durch Klassenräume und Gänge in Erwartung auf das kommende Christkind und das Schauspiel in der großen Halle. Für das Spiel wurden Kinder als Engel, Maria und Josef, der Esel und Ochse, den Hirten zugeteilt, nur das passende Jesuskind fehlte noch.\\\\
Dem blonden Mädchen hatte man die Rolle eines Engels zugewiesen, gleich hinter der Strohkrippe rechts neben Maria, im Nacken der mächtige Ochse und über ihr in luftiger Höhe gebastelte Papiersterne.\\\\
Nur die hübscheste und lieblichste Puppe war gut genug für das Jesuskind.}\\\\
Es war ihr Cicciobello.\\\\
Eifrig wurden am Abend vom Bruder die Haare meisterhaft zurechtgestutzt. Ihre Mutter strickte eiligst mit emsigen Fingern, die ganze Nacht, aus weißen Wollresten eine Hose, ein Jäckchen und eine Mütze. Aufmerksam wurden die Wackelbeine und Arme auf Hochglanz poliert. \\\\
Am Morgen, eilte das Mädchen erwartungsvoll in die Küche wo bereits das Feuer im Holzofen brannte, es roch nach warmer Milch und Kernseife. Die Glühbirne zuckte über dem Küchentisch, draußen war es noch dunkel. Auf der Fensterscheibe zur Balkontür hatte sich Dampf niedergelassen und zwischen Kannen und Gläsern saß auf der Ablage der kleinen schmutzig weiß lackierten Kredenz säuberlich lächelnd herausgeputzt das Jesuskind. \\\\
Nur die hübscheste und lieblichste Puppe war gut genug für das Jesuskind.\\\\
Nach der Aufführung in der großen Halle für Eltern und Kinder holte die Freundin den Cicciobello wieder ab.\\\\
Pech gehabt