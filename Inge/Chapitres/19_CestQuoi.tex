\chapter{Die Werkstatt}
%\vspace{\baselineskip}
\lettrine{S}{ie} hat mich wieder die Leselust, nach der Bergsucht, der Stricklust, der Nichtstunlust.\\\\
Wann ist sie erwacht die Lust am Lesen?\\\\
Es war dieses Buch im fahlen Sand grau mit den gestanzten Lettern die auf der ersten Seite des Einbandes tanzten in leuchtendem Rot – das „Dolomiten Sagenbuch“.\\\\
Ich fuhr mit meinen kindlichen Fingern durch die hauchdünnen Seiten und hörte die Stimmen der Elfen und Gnome, die aus smaragdgrünen Seen und dunklen Wäldern riefen.\\\\
\textit{Der Großvater hatte in seiner Schusterwerkstatt, mehr Zimmer als Arbeitsraum, eine große Holzkiste, die von der Schwere und Fülle vieler Bücher und deren Geschichten und Schicksale ächzte. Einmal als er außer Haus war, schlich das Mädchen, es war ihr nämlich verboten die Kammer zu betreten, in das Reich der Geheimnisse. Sie öffnete die sperrige Kistentruhe und wühlte tollpatschig und hastig, in Angst entdeckt zur werden in der Anhäufung von schmalen, dicken, großen, kleinen, schwarzen, bunten und illustrierten Büchern. Sie konnte erst kaum lesen, die Buchstaben schwirrten und sperrten sich noch aufgespürt zur werden. Und da lag es unter alle den Büchern, die schon viele Reisen mitgemacht hatten von Südtirol nach Österreich und von Österreich nach Südtirol, auf luftigen Karren und stickigen Zügen, im Krieg und Frieden, das Buch der Sagen. Ehrfurchtvoll die Prinzessinnen und Prinzen in den Seiten des Buches gefangen, steckte das Mädchen das Buch unter den Kitteln und verschwand aus dem Zimmer. \\\\
So lag es dann da unter dem Kopfkissen oder Matratze, gehortet der Schatz aus dem Reich der Sagen, um von dem Mädchen wach gemacht zur werden. \\\\
Des Nachts wenn es im Haus ruhig wurde, nahm sie es heimlich hervor strich mit ihren Kinderhänden über das Seidenpapier und sah die verzauberten Frauen, die mutigen Burschen und bösen Zwerge und hoffte, dass Großvater das Buch nie vermissen würde.\\\\
Wenn das Mädchen mit seinen drei Brüdern, dem Vater und der Mutter mühsam Berge hochging, durch dichte Wälder streifte, an Weihern und Seen harrte, seine nackten Füße auf Moos wanderten, da sprach sie zu den Steinen im Flussbett, spähte in Löcher und unter Wurzeln, lauschte der Stille und vernahm manchen Hauch des Streichelns auf ihrer Wange.} \\\\
Aber davon erzählte sie niemandem. 
