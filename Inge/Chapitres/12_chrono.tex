\chapter{Das Paradies}
%\vspace{\baselineskip}
\lettrine{D}{ie} Erinnerungen der Kinderzeit sind wie ein blühender Garten - eine kurze Zeit nur - bis man dann von den Erwachsenen wie Adam und Eva aus dem Paradies vertrieben wird. \\\\
\textit{Der Vater hockt am Küchentisch, auf der harten Eckbank und stützt seine Ellenbogen auf die Tischplatte, die Handflächen umschließen dabei seine Ohren, damit er das Lärmen der Kinder, die in der Wohnküche spielen, nicht hört. Die aufgeschlagenen Seiten des Buches liegen vor ihm. Eine Stubenfliege summt um seine Nasenspitze, setzt sich dann schließlich auf das Papier. Seine Ehefrau im Schürzenkleid schöpft aus dem dampfenden Kochtopf die Speckknödel in den abgestellten Teller, gießt Wein nach und dreht sich weg. Sie kennen sein Schweigen. Beim Essen und Lesen soll Ruhe sein. In ihren Ohren dröhnt die laute Stille. Überhaupt redete er nicht viel, außer er war bei der „Tresl“, ein beengter Ausschank in der Schutzengelgasse. Auf den Baustellen wo er arbeitet pfeift er immer ein Lied.\\\\
Es war vielleicht einer jener Tage im Sommer oder auch am Ende des Jahres 1939, wo der Himmel vielleicht stand im einem beständigen blau, da zog seine Familie mit den wenigen Koffern, in denen zwischen Hemden, Hosen und Socken, auch die Hoffnung und der Zweifel ruhten, über den Brennerpass nach Österreich. Die Mutter noch eine außergewöhnlich stolze schöne Frau mit tiefschwarzem Haar und tiefschwarzen Augen, an beiden Händen führte sie einen blonden lockigen sechsjährigen Jungen und seine jüngere Schwester.\\\\
Der Vater zog dann sofort in den Krieg.\\\\}
Im Haus 6o Jahre später, zurück in der alten Heimat, sitzt der Großvater mit seiner Frau und der Tochter am Tisch. Im Wohnzimmer, dessen langen Seitenwände vom Boden bis zur Decke voll sind mit Büchern, spielen die Enkelkinder und durch die geöffnete Terrassentür dringt süßer Frühlingsduft in das Zimmer, eine Amsel ruft auf dem blühenden Holunderbaum. Da fragt der Enkelsohn, ohne Vorwarnung ins Nichts: Opa, wie war das bei dir im 2. Weltkrieg? Mit stockendem Blick schauen die Mutter und Tochter auf ihr Besteck und stochern verhalten in ihren Tellern weiter. Sie warten.\\\\
Darauf, dass der Großvater stumm aufsteht und geht.\\\\
Diesmal war es anders.\\\\
Mehr und mehr erzählt er von sich als Kind: Vom Hunger, der in die Stube kam, dem Leid im kalten Winter, den anderen Kindern in den braunen Uniformen und ihnen, den ewigen Fremden. \\\\
Großvater rückt den Stuhl leise beiseite, streichelt sanft über die Stuhllehne und schaut schmunzelnd zu seinem Enkel.\\\\
„Stell dir vor, man würde dir heute erzählen, das mit Gott wäre alles eine Lüge.“\\\\
Kein Tag war für mich jemals so klar gewesen wie dieser. \\\\
Vielleicht fragt mein Vater sich deshalb oft, wieso selbst die Luft, die er atmet, fremd geblieben war.\\\\

Vatti ich liebe dich