\chapter{Der Keller}
%\vspace{\baselineskip}
\lettrine{D}{ie} Wohnung meiner Eltern, sie lag im zweiten Stock eines kleinen, unscheinbaren Hauses, gebaut in den 60er Jahren. Sie hatte eine Wohnküche, zwei Schlafzimmer, eine Werkstatt, ein kleines Bad mit Holzofen und einen Abstellraum. Die Wände des Hauses waren grau getüncht mit grünen Jalousien. Es gab zwei Balkone, eingerahmt von Eisengeländern, wo wir vier Kinder uns heimlich bekämpften, wer wohl der dümmste sei der es wagt seinen Kopf durch die Gitter zu stecken, was dann aber keiner machte.\\\\
Die weißen Bettlaken blähten sich im Sommer durch den Wind auf, welche auf den Wäscheleinen hingen, die sorgfältig in Reihen vor den Balkonen gespannt waren. Taschentücher, Hosen, Hemden, Kindersocken, jeden Tag.\\\\
Im Winter war es kalt. Auf den Scheiben der Fenster und Außentüren wuchsen wie aus Zauberhand  Eisblumen und zarte Sterne, während sich im Sommer die Hitze erbarmungslos unsere Leiber suchte - durch Fensterritzen, Türen, ein schlecht isoliertes Dach, um sich dann rücksichtslos wie ein feuchtes Tuch auf unsere Gesichter, Köpfe und Glieder niederzulassen. \\\\
In der Küche kochte die Mutter auf dem Holzherd in großen Emailletöpfen Marmelade. Das Holz knisterte im weißen Herd und gestreifte Bienen, mutige Wespen, kreisten, summend und freudig in belohnender Geschäftigkeit ihre Bahnen, angezogen vom süßen Duft der Früchte. Der Holzboden aus gebürsteten Dielen kratze und schierte auf unseren nackten Kindersohlen. Die Mutter rührte mit dem Kochlöffel im Topf und die Schweißtropfen auf ihrer Stirn benetzten das schöne Gesicht, eingerahmt von dichten, kräftigen, dunklen Haaren. Die Iris ihrer Augen suchten sich einen Wettstreit mit dem Glitzern der Marmeladeblasen im Geschirr. \\\\
Im Erdgeschoss des Hauses befand sich ein Keller. Drei Stufen aus Stein führten auf lehmigen, erdigen Boden, der kalt war, und bis über die Beine hochkroch, im Sommer wie im Winter. Es roch nach Holz und Moder und Katzenpisse. Dort stand eine ärmliche Kommode aus Fichtenholz, mit drei großen Schubladen und sechs Eisenbeschlägen. Auf ihr lagen, Bretter, Planken, Latten, und Leisten wild durcheinander, Schaufel, Hacken aus Holz und Stahl, ein ergrauter Stuhl, eine Rodel, ein Rechen, Blech und viele Schuhe fügten sich zu einer großen Familie und ganz oben, auf wackeligen Beinen stand ein Vogelkäfig aus Holz. \\\\
Ich sah die Leere im Vogelkäfig, die dünnen Holzstäbe, das kaputte Türchen, die Spinnweben die sich anstelle des Vogels häuslich eingerichtet hatten und hörte den Gesang der Nachtigall.

