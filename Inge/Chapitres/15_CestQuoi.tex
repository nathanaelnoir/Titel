\chapter{Der verbotene Garten}
%\vspace{\baselineskip}
\lettrine{A}{ls} wir uns entschieden auf eine der Birken in seinem Garten zu klettern, Ast für Ast, mit schlotternden Knien, dem Wipfel entgegen, da begann vielleicht unsere Freundschaft. \\\\
Ich erinnere mich an das Zittern der Birkenblätter und das glänzende Weiß der Baumrinde, welche sich an manchen Stellen schälte wie alte, verbrannte Haut und dann aufrollte wie Hobelspäne. Wir betrachteten die von den Wolken gefilterten Lichter der Sonne, wippten in der Baumkrone und drückten die Nasen in den Himmel. Unter uns lag der Grasteppich, in einem satten Grün. Der Rasen war kurzgemäht, die Sandkiste neben der Buchenhecke aufgeräumt. In quadratischen Beeten wuchsen Rosen, deren Wohlgeruch durch den Garten strömte, vermischt mit dem schweren Aroma des Buchsbaumes. Nichts war in dieser Anlage dem Zufall überlassen. \\\\
Wir spielten manchmal im Garten, inmitten der Nadelbäume die in einer Ecke eine Gruppe bildeten, oder versteckten uns unter den blühenden Büschen des Ballonstrauches und der Prachtspieren, lauschten dem Wind der mit sanften Finger durch das Laub strich, hie und da flog eines der Blütenblätter lautlos auf die Erde.\\\\
Er hockte mit seinen dünnen Beinen und Armen im Schneidersitz neben mir und wir steckten die reifen Kirschen hinter die Ohren und in den Mund. Barfuß liefen wir durch das weiche Gras, unsere Fußsohlen färbten sich grün, hüpften über die gefegten warmen Steinplatten entlang der angelegten Wege, und naschten heimlich von den roten Ribiseln und sauren Stachelbeeren. An den Tagen, wo die Nachmittagssonne niederbrannte schlichen wir in den hinteren Garten über die Steinstufen, wo wir im Schatten einer großen Trauerweide die Hände zu einer Faust formten und den Eulenruf der älteren Geschwister übten. Oft strichen wir an den Kellerfenstern vorbei, die mit Eisenstäben vergittert waren und von Efeu berankt und genossen die kühle Luft. \\\\
\textbf{Hier haben wir geschwiegen und geträumt.}\\\\
Unter dem Garten, abgegrenzt von einer hohen Steinmauer lag terassenförmig vergessen eine Wiese. Zwischen den aufgelassen Weinrebstöcken streckten die Schafgarbe, Vergissmeinnicht und die Witwenblumen ihre Köpfchen in die wärmende Sonne. Brennnessel in dichten Büschen wuchsen wild angelehnt an der Steinmauer, neigten sich im Windhauch. Die Mauereidechsen huschten in ihre Verstecke durch die Ritzen. Wiesenmargeriten und verblühter Löwenzahn zauberten eine Lustigkeit zwischen die verdorrten Gräser und Halme. Eine tote Kröte ruhte hinter einem Sandhügel. Das summente Murren der Bienen die in den hohen Gräsern hin und her taumelten, oder die Blütentrichter der Blumen umkreisten, vermischten sich mit dem fröhlichen Geplaudere. Ab und zu lockte eine Amsel mit ihrem Gesang. Der Sommerduft ließ die Dinge heiter erscheinen. Selbst regnerischen Tage vermochten es nicht unsere gute Laune zu verderben. Die Landschaft ringsum wurde unsere. Man ließ der Natur ihren freien Lauf.\\\\
\textbf{Hier haben wir gelacht und geträumt.}\\\\

Wenn der kommende Abend das blaue Nachtuch über das Tal legte und im Talkessel die Lichter durch die Häuserfenster und auf den Straßen blinken, rannten sie beide nach Hause, er in das große weiße Haus, sie in die Arme ihrer Familie. 