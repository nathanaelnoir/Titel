\chapter{Der Soldatenfriedhof}
\chapterprecishere{``Zwischen Geburt und Tod müssen wir Durchhalten"\par\raggedleft--- \textup{Fred Amonn}}
\lettrine{W}{ieder} einer dieser schlaflosen Nächte, wo die Schmerzen nach dem Unfall durch den Körper strömten, die Nerven summten wie eine Maschinerie. Das Bettlaken klebte auf der Haut. Die Gedanken der Nacht lagen schwer auf den Kopfkissen, nur das vertraute Brummen des Kühlschrankes beruhigte sie. Die Balkontür stand offen, im Lichtschein einer Straßenlaterne tanzten in Scharen Mücken, irgendwo sang eine Amsel. Der Lärm des Verkehrs auf der Straße hinter den Häuserzeilen versank in der Dunkelheit der Nacht gleich dem eintönigen Murmeln eines Rosenkranzgebetes. Sie war müde, wartete auf die Helligkeit des Tageslichtes, damit sie endlich gehen konnte. Gehen, um zu vergessen, gehen um nicht sterben zu wollen. \\\\
Hinter dem Krankenhaus führte eine asphaltierte Straße hoch zum alten Soldatenfriedhof. Beide Arme umfassten die Krücken. Sie kämpfte sich voran, erschöpft hielt sie immer wieder inne, humpelte weiter, keuchte und dann konnte sie die Tränen nicht mehr zurückhalten. Du, hatte er gesagt, es ist vorbei. Wochen waren seitdem vergangen.\\\\
Ganz leicht und sanft flossen die Schweißtropfen über die Stirn. Sie schniefte und rotzte, wischte den Rinnsal mit dem Ärmel weg, Staubkörnen fielen auf ihr Gesicht und vermischten sich mit den Tränen auf ihren Wangen. Zwischen den hochgewaschenen Pappeln an beiden Seiten zum Eingang des Friedhofes war sie ganz allein. Sie fühlte sich einsam und verloren. Am liebsten wäre sie umgekehrt. Aber bis zu den Bänken zwischen den Gräbern war es nicht mehr weit. Von den Apfelbäumen in der darunterliegenden Wiese hörte sie Vogelstimmen, die zwischen den Ästen aufschreckten. Ein Güterzug hinter dem hochstehenden tiefgrünen Maisfeld rauschte laut vorbei.\\\\ 
Der Friedhof stand auf einer Anhöhe von leblosen Heckenbüschen umgeben, auf den Steinmauern huschten Eidechsen durch Hohlräume und Klüfte. Der Himmel an diesem frühen Morgen erwies sich noch gnädig, weiße Federwolken wirbelten am Himmel. Im Licht der aufgehenden Sonne zeigten die Stämme und knorrigen Äste lange Schatten. Die Kieselsteine knirschten unter den Schuhsohlen. Das Eisentor klemmte, das Quietschen ließ sie aufschrecken.\\\\
Die Geister der Toten begrüßten sie.\\\\
Der Sommer in diesem Jahr glühte und schnaufte von den hohen Temperaturen, die Tage lagen in schwüler Betäubung, in den Nächten donnerte das Gewitter von den Bergspitzen durch die Täler, nur ein flüchtiger Auftritt, der kaum Abkühlung verschaffte. \\\\
Mit Kies belegte Wege durchwanderten die schlichten Gräberkuppen, deren Kopfseite einheitliche Steinkreuze dekorierten. Auf den Beeten wuchs Efeu, selbst dieser stöhnte und der andauernden Hitze. Die Spitzen der Blätter dursteten nach dem Herbst. Über Stufen wand sich um ein Denkmal ein gesetzter Weg zu einer Kapelle. Sie wirkte verlassen. Verlassen wie die zahlreichen, eingemeißelten Namen der Toten, die Bilder der jungen Männer, aufgereiht auf einer bogenförmigen Wand an der Hinterseite des Hauses. Erstarrt blickten die Gesichter der Vergangenheit aus den Nischen. Sie trat aus dem Gebäude, hielt an, an der Holzbank unter den zwei Tannen blickte zu den Spitzen der Bäume und weiter über die Reihen von Grabstätten und weinte dann bitterlich.\\\\
Im Klageschrei wollte sie es ihnen erzählen, den Toten, den Antons, den Karls, den Kriechtieren und den Ameisen auf dem erdigen Boden. In der schweigenden Menschenferne hat sie sich gefragt, nach dem Wieso. Sie wird seiner gedenken in der Erinnerung.\\\\
Beim Verlassen des Friedhofes, schloss sie das schwere Tor und sie konnte sie nicht mehr hören, die Stimmen der Verstorbenen.\\\\

Morgen wird wieder ein sonniger Tag sein. 