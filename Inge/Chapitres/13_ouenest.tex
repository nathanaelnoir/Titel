\chapter{Das Rotkehlchen}
%\vspace{\baselineskip}
\lettrine{D}{er} Gemüsegarten vor dem Haus meiner Mutter gereift von der Sommersonne leuchtete im Abendlicht. Blattsalat, Tomaten, Erbsen und das Gras der Karotten standen im Schönheitswettbewerb für die Herbstschau. Das Wasser im Steintrog unter der Steintreppe, der zur Bewässerung diente, formte schwingende Kreise, welche der stetige, langsame Tropfenfall im tock, tock erzeugte.\\\\
Geduldig warteten die Gießkannen aus verbeultem Zink.\\\\
Sorgsam angelegte Wege zwischen den Beeten, gestampft durch fleißige Arbeit zogen sich wie Gittterlinien durch den erdigen Boden. Eine strubbelige schwarze Katze wärmte sich auf der Gartenmauer unter dem Birnenbaum. \\\\
Meine Mutter wischte sich eine Haarsträhne aus der Stirn, ein verrichtetes Lächeln huschte über das Gesicht, dann streifte sie die schmutzigen Hände an der Schürze ab und nahm den Korb gefüllt mit fruchtigen Tomaten. \\\\
Dahinter im ersten Stock wohnten die Hausfrau, die Vermieterin und ihr Mann. Ihre gezopften grauen Haare hatte sie zu einem Kranz hochgebunden, so wie es damals noch viele Frauen taten. Die vom Lande und manche noch in der Stadt. Der sonnengegerbte Körper war in einfache Arbeiterkleider gehüllt, ein karierter Rock hing über ihre geschundenen Waden, sie trug eine geflickte Bluse, darüber hatte sie eine Schürze gebunden. Eine verlebte Frau von kleiner Gestalt, auf kurzen Beinen und mit männlichen Händen. Die „Zilli“, von uns Kindern und Mutter gefürchtet, gehasst. \\\\
Sie verbrachte die Tage damit meiner Mutter und uns Kindern das Leben schwer zu machen. Da stand sie oft tagelang am Fenster hinter den Gardinen, am Gartenzaun, auf dem Balkon, auf den Treppen zum Eingang, um uns Kinder und Mutter anzupöbeln und zu erschrecken. Den Hund Rolfi auf uns zu hetzen war noch das kleinste Übel. \\\\
Am Berg auf der anderen Seite des Tales, der „Königanger“ verabschiedete sich die Sonne mit den letzten Strahlen des Tages. Sie grüßte zum Abschied den Bergkamm, hier drüben auf der anderen Seite des Tales kitzelte sie noch die Blätter mit leisen Wind. Drüben legte sich der Schatten über die Nadelbäume und Wiesen zur Abendruhe.\\\\
Im Gemüsegarten meiner Mutter da hing dann am nächsten Morgen auf einem Holzstab ein Rotkehlchen an einem Strick. Den Kopf mit den starren, geöffneten Augen nach unten, der rötliche Bauch gebleicht im Federkleid, die Krallen der zarten Beinchen formten ein klägliches V, benutzt als Vogelscheuche, gewarnt.\\\\
Jetzt hätte ich gerne einen Spaten zur Hand.
