\chapter{Vom Fliegen}

\lettrine{B}{ruchstücke} der Erinnerung. \\
Eine Wiese. Das Vergissmeinnicht in der abgebrannten Ruine, der Kirschbaum im Nachbargarten, süße Stachelbeeren und eine Kinderhand auf der Lenkstange eines Fahrrades.\\\\ 
\textit{Der Vater fuhr mit seiner kleinen Tochter auf einem schwarzen Herrenrad zum See. Sie konnte bereits  gut laufen auf ihren  tollpatschigen krummen Beinen. Er hatte sie in einen Kindersitz vor die Lenkstange des Fahrrades gesetzt. Damals noch mehr eine Aufhängung aus Gedrähten als ein sicherer Sitz. Lustig baumelten die Füßchen im Wind. Erst beide Beinchen nach oben , dann beide Beinchen nach unten. Seine geübten kräftigen Waden radelten den Vater an diesem Sommertag zum Vahrnersee, der sich oberhalb der Stadt befand. Die braunen Augen blitzen, das blonde schulterlange Haar klebte von Schweiß getränkt im Nacken. Zuerst eine Steigung die Brennerstraße hoch, auf trockenen Asphalt, dann nach einer Abzweigung auf einem Schotterweg hinunter zum See, durch einen Eisenbahntunnel aus Stein, weiter entlang einen holprigen Weg, gepflastert von Steinen, Blättern und Wurzeln, den See auf der linken Seite im Blick. \\\\
Am östlichen Ufer machte er Halt, fernab von den Ausflüglern auf der anderen Seite, die sich auf der Liegewiese in der Sonne aalten. Er nahm das Kind vom Rad, stieg aus der kurzen Hose, legte das Hemd zur Seite und nahm das Mädchen an der Hand. \\\\
Mit dem Kind auf dem Rücken sitzend glitt er durch das Wasser, fügte seine Arme ruhig in gleichmäßigen Bewegungen von der Brust hin und zurück, die Füße taten es ihnen nach, bis hin zur tiefsten Stelle im See. Im Süden des Sees wiegte sich das Schilf. Das Wasser war ruhig, schlug nur leichte Wellen im lauen Grün. Nur nicht loslassen, vertrauen, keine Angst.\\\\
Es ist nun Abend, es ist heiß und die Straßen in den 60iger Jahren sind noch leer. Das Mädchen juchzt und jauchzt vor Freude. Der Fahrtwind zieht an den blonden Locken, das bunte Röckchen fliegt und flattert bei der Abfahrt nach Hause.}\\\\
Du hast mir Flügel geschenkt. Danke Vatti