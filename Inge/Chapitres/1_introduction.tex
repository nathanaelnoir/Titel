\chapter{„Vergelts Gott“}
%\vspace{\baselineskip}
\chapterprecishere{``
Wenn dich die Welt aus ihren Toren stößt, 
So gehe ruhig fort, und laß das Klagen.
 Sie hat durch die Verstoßung dich erlöst
 und ihre Schuld an dir nun selbst zu tragen."\par\raggedleft--- \textup{Karl May, Im Reiche des silbernen Löwen}}
\lettrine{D}{er} kleine Hans durchstrich wie ein verletztes Tier das Gehölz. Mit vorsichtigen Schritten befühlten seine nackten Füße einmal das weichfeuchte Moos, dann die Nadelflecken unterhalb der Waldungen des auffallenden Losers nördlich des großen Sees in Altaussee. Hier ging er selbst für sich. Der Wald mit dem tiefgrünen Ästendach und den mächtigen Kronen verdunkelte das Blau des Himmels. Lichterfetzen beleuchteten die Blätter des Farn, den verblühten Klee und den Ameisenhaufen unter einem Tannenbaum. Es war Montag. Die kurze Hose scheuerte an den Beinen. Seine rechte Hand hielt das blutverschmierte Taschentuch.\\\\
Wild war er losgelaufen, mit dem Gefühl von Steinen in der Magengrube, einen Trampelpfad den Berg hinauf in den Wald. Er hatte sich hochgekämpft, rutschte aus, stapfte weiter, immer weiter. Die Erschöpfung ließ seine Glieder schwer werden. 'Net dosiger' nannten ihn die Jungen in der Klasse. Sie machten sich einen Spaß daraus, ihn auszuspotten, abzufangen und zu verprügeln. Am Vortag hatte der Rastl ihn an die Wand gedrückt. Der „Sitzenbleiber“, ein aufgedunsener Junge, einen Kopf größer als er. Rastl tauchte auf wie eine angeschwollene Wasserleiche, kurze Haare aus gelben Draht, mit einem kugelrunden Kopf und bösen wasserblauen Augen. Jetzt würden sie es nicht mehr tun. Ihn quälen, verspotten. Den Bub aus Südtirol. \\\\
\textit{Ausländer, Fremder ... }\\\\
Diese Worte trugen in hoch, höher vorbei am Ufer des Sees, durch die Wiesen, wo der Löwenzahn verblüht die Köpfchen reckte, ohne Gruß. Er stob vorbei an Wegrandmargariten, den Föhren und Fichten, über Wurzelstöcke und totes Astwerk, Armen und Füßen gleich die nach ihm traten, griffen und zuschlugen. \\\\
In der Lichtung steckte er das Taschentuch in die Hosentasche und die Hügel und Matten von glänzenden Blattwerk wirkten wie schlafende urzeitliche Tiere. Wind kam auf, fuhr in die Äste, als er sich im Wald unter den schützenden Zweigen verkroch. Dort knöpfte er sein Hemd zu, zornig und tränenblind. Alles herum war tiefer Ernst. \\
Jetzt wo er allein war, wurde alles groß und freundlich und begann ihm zu gehören, sein Wald, sein Weg, seine Bäume, seine Geräusche. Er fing an die Vögel zu benennen, mit den Ameisen zu reden, Löcher in die Erde zu bohren. Er befühlte den rissigen Stein mit allen zehn Fingern, roch am Farn, den er in den Handflächen zerrieb. Kroch auf allen Vieren durch den erdigen Boden, einem grünschimmernden Käfer hinterher. Schaute durch die Äste und blinzelte mit den Augen, trotzig verliebt in die Sonnenstrahlen.\\\\
Er hörte noch das Aufschreien im Klassenzimmer und das Schweigen danach. Dann spuckte er auf seine Wunden an der Hand und dem Ellenbogen. Die Müdigkeit nahm neben ihm Platz, berührte die pochenden Schläfen, die zittrigen Hände und Hans schloss seine ermatteten Augen. \\
An jenem Morgen hatte alles wie sonst begonnen. Still und schweigsam hockte er in der letzten Bank, der Neue. Auf seinen beiden Schenkeln versteckt unter der Schulbank lag aufgeschlagen das Buch von Karl May. Die Knaben übten das ABC, er flog durch die Zeilen und ritt mit den Männern durch den wilden Westen, da wo sich die Berge erhoben und Graslandschaften ausbreiteten, keine Bäume und Sträucher wuchsen, die Sonne gnadenlos auf die Prärie knallte. \\
In der Pause, alles wiederholt der spöttische Blick, das Schubsen und Stoßen. Wie der Rastl dabei reinbiss in das weiche Brot. Da packte Hans ihn mit beiden Händen am Hals zog ihn zum Fenster im zweiten Stock des Schulhauses. Wimmernd hing er zuletzt über dem Fensterbrett. Er schlug zu, schlug ins Gesicht, trommelte mit seinen Fäusten rasend auf ihn ein. Dann war es plötzlich leise und es roch betäubend aus Millionen von unscheinbaren, honigsüßen Fliederblüten und in der Stille und dem Duft fällt der Junge zusammen, wie eine Puppe, aus der Sägemehl rann. Das Brot lag daneben auf dem Fußboden, die Wurstscheibe hingeblättert am Stuhlbein. \\\\
Stunden vergingen, Stunden im Dickicht des Waldes, Stunden in denen er wegdöste und sein Kopf auf die Knie sank und wenn er hochschreckte durch den Ruf eines Vogels oder das Rauschen in den Blättern, waren sie noch immer da, die schrecklichen Kinder, so stetig wie das Pfeifen des Windes. \\\\
Hans benützte für den Heimweg den Abkürzungsweg durch die Felder, blieb stehen, hob seine Hand und grüßte die kahle, schroffe Trisselwand. Er lief den Zickzackweg hinab und möchte erleichtert sein, noch einmal davongekommen. Stolpert zwischen Wolfsmilchstauden und Dornenbüschen. In den Wiesen zirpten die Grillen wie toll. Am Seeufer angekommen, wusch sich Hans das Gesicht und den Hals mit dem Seewasser, säuberte das blutige Taschentuch. Aus der Spiegelung des Sees sah ihn ein kleiner dünner Junge an. Er sog die feuchte Luft an und die bedrohlichen Gesichter verschwanden, eins nach dem anderen, lösten sich im glasklaren Wasser. Niemand war am Ufer. Wie ein unverdienter Segen schien ihm die Einsamkeit. Er dachte nach, erzählen, auf keinen Fall, kein Wort und schlenderte dann nach Hause, die Holztreppe lautlos hoch, während oben auf dem Berg der Himmel rot leuchtende Färbungen legte. So viel Zeit ist über all dem vergangen, dass die Sonne bereits schräg über den Bergen stand und der Junge im Gehen lange Schatten warf. \\\\
Eine schweifende Seele begleitete ihn. Hinter den Grenzen tobte zur gleichen Zeit ein schrecklicher Krieg. 





