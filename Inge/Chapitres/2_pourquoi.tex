\chapter{Alain O.}


\lettrine{I}{n} einer Kleinstadt kennt jeder jeden, und jeder niemand. Eigentlich.\\
Die Villa stand auf einem Hügel, am sonnigen Osthang der Stadt, mit der vornehmen Aussicht auf die Stadt dort unten. Sie war von gepflegten Bäumen und einem hohen Zaun vor neugierigen Blicken geschützt. Zu Füßen lag der Dom mit seinen zwei Türmen, der Pfarrkirche, Laubenhäuser, die Hofburg, und Gassen, geschäftiges Treiben bei Tag, ein Lichtermeer in der Nacht. Die Kinder in den dürftigen Häusern dursteten nach der Welt der Reichen und phantasierten durch ihre Räume. Sie erzählten sich Geschichten von den Menschen die dort lebten. \\\\
\textit{Es ist März. \\\\
Der Bruder geht mit seiner Schwester durch die Stadt. Der Winter liegt noch schwer auf den Bergen, dieses Jahr gab es viel Schnee. Die ersten Leberblümchen quälen sich durch das Laub.\\\\
Eigentlich wollen sie an diesem Sonntag nur einen Kaffee trinken, der Bruder ein Bier. Da bringt sie der Weg über die Brücke des Rienzbaches. Mal schauen. Es geht sich doch gut. Die Sträucher und Laubbäume wurzeln noch grau im Frühlingschlaf, es nieselt leicht, im Fluss tummeln sich zwei Mandarinenten und eine emsige Wasseramsel.\\\\
Nach der Finanzkrise um 2010 boomt jetzt der Baumarkt. Krahn um Krahn reiht sich in die Landschaft. Neue Wohnsiedlungen, prägen die Umgebung. Wo sich einst Felder und Weingärten zusammenfügten, halten sich jetzt Dach und Dach, Balkon und Balkon, Wohnungen und Reihenhäuser verflochten die Hand.\\\\
Sie kommen an der alten Fabrik vorbei, wo seinerzeit Bier gebraut wurde und wandern dann weiter hoch zur Promenade. \\\\
Das Haus weilt versteckt auf einem Hügel, verwachsene Sträucher und Bäume versperren den Blick. Wo damals die Einfahrt zum Haus war, da liegen jetzt im Dickicht Dosen, Plastikflaschen und grüne Hundekotsäcken zwischen vermoderten Ästen und Gestrüpp. \\\\
Auf einem Zaun verweist ein Verbotsschild den Eintritt zum Areal. Die beiden Geschwister klettern über den Maschenzaun und bahnen sich einen Weg durch Rosendornen und Unterholz zum Haus. \\\\
Es ist ein einstöckiger Bau aus den 60iger Jahren in einem müden Weiß und mit verblassten blauen Holzjalousien. Der Weg zum Eingang ist gepflastert mit Porphyrplatten. Auf der rechten Seite befindet sich als Anbau eine Garage mit einem himmelblauen Tor. Unter Astwerk und wilden Rosengebüsch harrt eine Hundehütte, ein kleines Häuschen aus Holz. Die Haustür, ein Monstrum mit Eisengitter und schweren Glas bleibt verschlossen. Ein dunkelgemusterter brauner Vorhang hängt schwer über dem Fenstern. Ein Schild mit Eintritt verboten. Ein zweites Mal.\\\\
Die beiden suchen sich einen Weg hinter das Haus. Der erste Augenschein fällt auf einen runden Swimmingpool vor einer Steinterrasse mit noblem Rundblick, und zu deren Fuß im Tal wartet die Stadt. Lange schon hat hier keiner mehr seinen Körper trainiert, eine wackelige, rostige Eisenstiege führt in die Tiefe. \\\\
Zwischen den vertrockneten hohen Gräsern des letzten Sommers recken vereinzelt Tulpenblätter ihre Nasen in die wärmende Luft. In der Ecke der Terrasse ruht aus Stein gebaut ein offener Kamin. Große Fenster bis auf den Boden und Türen führen auf die Veranda.\\\\
\textbf{Wie oft saßst du hier draußen Alain, versunken in dir?}\\\\
Da finden die beiden eine offene Tür, ein Spalt nur, welche bereits von anderen Neugierigen aufgebrochen scheint. Sie treten in den ersten Raum, der Boden ist sorgfältig aus Holzquadraten gelegt. Es ist erstaunlich sauber, nahezu aufgeräumt, der Raum ist praktisch leer. Nur Bücher, Jugendbücher können sie erkennen und Zeitschriften - „Der Kosmos“ -  stapeln sich in einem Winkel. Die Wände sind weiß gestrichen. Sie gehen weiter, nur der Schein des Lichtes der durch das Fenster dringt erhellt die Räumlichkeiten. Es öffnet sich ein weiteres Zimmer mit einer großblättrigen verzierten Tapete, an deren Wand lehnt verlassen ein Infusionständer. Sie betreten den Gang. Auf der rechten Seite treffen sie auf ein großes Badezimmer mit rosa glänzenden Fliesen, eine Badewanne, eine Toilette, ein Bidet, dann ein weiteres Bad in biederem Weiß, mit Dusche und Waschbecken.\\\\
Das Wohnzimmer ist groß. Hier stehen einsam, zwei große mächtige Schränke aus dunklen Massivholz. \\\\}
\textit{\textbf{Wie oft saßst du hier drinnen Alain, versunken in dir?}
\\\\Im Flur wird es finster. Seltsam beklemmend scheinen die einzigen Bilder an der Wand. Zeichnungen, Malereien aus Kinderhand – ein einzelner Baum, ein kleiner Prinz vor einem Schloss, heilige Bilder - mit barmherziger Mutter und Sohn. Eines mit einem Kirchenaltar. Es knirscht unter den Schuhsohlen, von kaputtem Glas. Auf einer Schukommode liegen Dominosteine.
Es ist friedlich. Die Geschwister setzten sich auf die kniehohe Mauer der Loggia,rauchen eine Zigarette und schauen hinunter zu ihrer Stadt. Jeder in seinen Gedanken.
Bald kommen die großen Baumaschinen und reißen alles nieder.}
\\\\
Alain war ein bildschöner Junge, zwei Jahre jünger als ich, mit dunklen glatten Haaren, die zu einem Rossschwanz gebunden waren, und einem bleichen Gesicht. An einem Wintertag sah ich dich unter den großen Lauben. Du trugst einen langen schwarzen Mantel und schwere Stiefel. Du hast mich nicht gekannt. 
Alain war krank. Er war ein Bluter. Er erschoss sich mit 33 Jahren am 01.05.2000 in einem der Zimmer. 

