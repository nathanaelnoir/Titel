\chapter{Der Weihnachtsmarkt}
\chapterprecishere{``Die Vergangenheit ist nicht tot,
sie ist nicht einmal vergangen"\par\raggedleft--- \textup{William Faulkne}}
\lettrine{J}{ahrzehnte} sind inzwischen vergangen, und doch habe ich jenen gepflasterten Platz gepudert mit einer hauchdünnen Schneeschicht noch immer deutlich vor Augen. Nach mehreren Tagen mit leichtem Schneefall, haftete die Kälte an den Baumästen und Giebeldächern. Der Nachmittag hatte weich begonnen, trübes Licht lag auf dem Tal. Die Kälte im Hochtal zitterte schneearm. Von den  Weihnachtsständen auf der gegenüberliegenden Seite des Klosters, erklangen die üblichen kitschigen Weihnachtlieder. Ein kalter Windstoß strich durch die Passage, Kinderlachen ertönte aus der Ferne. Filzpatschen und handgestrickte Wollsocken baumelten an der aus Holzbrettern genagelten Verkaufsbude. Der süße Duft von gebrannten Mandeln und Glühwein rauschte, wie ein Weihnachtsengel über die Köpfe der Besucher. Damals habe ich alldem keine Interesse geschenkt. Niemals hätte ich gedacht, dass sie einen solchen Eindruck hinterlassen würden, und schon gar nicht, dass ich mich nach Jahrzehnten noch bis in jede Einzelheit an sie erinnern würde. An jenem Tag war mir dies alles vollkommen egal. Mit den beiden Kindern an der Hand stand ich auf jenem Platz. Nichts war damals von Bedeutung. Ich dachte nur an mich, an den Mann, der da gerade auf mich zukam mit zwei großen vollgepackten Einkaufstüten. Wie ein Hammerschlag kehrt die Erinnerung zurück an die weißen, dicken Federbetten. An das, was ich in diesem Moment dachte, was ich fühlte. Ich war verliebt. Ich liebte. \\\\
Die Lage war kompliziert. \\\\
Es begann an einem Nachmittag, als ich mit meinen beiden Kindern, ein Junge von sieben und ein Mädchen von sechs Jahren mit ihm nach Bruneck fuhr, kurz vor Weihnachten. Das erste Mal, dass wir einen Ausflug gemeinsam unternahmen, mit der Hoffnung auf den Vorder- und Hintersitzen an einen möglichen gemeinsamen Neubeginn. \\\\
Die geheimen Tage und Nächte waren bisher nur die ihren gewesen.\\\\ 
Im Inneren des Wagens war es still, die Kinder auf der Rückbank saßen berauscht von der Vorfreude des Kommenden. Aus dem Radio dödelte belanglose Musik. Sorgfältig war der Ausflug vorbereitet mit einem Besuch einer Eisenbahnausstellung und einer Filmvorführung des neuesten Film von „Harry Potter“. Für dieses Schauspiel hätten die Kinder ein Jahr auf Süßigkeiten verzichtet. Sie gingen mit verstohlenen Köpfen durch die Gassen der Stadt, als hätten sie etwas verloren. Die aufgezwungene Freundlichkeit schmerzte. \\\\
Wenn ich heute zurückdenke, kommen  mir als erstes die Plastiktaschen in den Sinn. Das Kindergelächter, der Duft von Weihnachten. Ganz deutlich, so deutlich, als könnte ich die Zimtsterne, die Kerzen, den heißen Glühwein berühren. In all dem gibt es nur zwei Menschen. Der Mann der auf mich zugeht und mich. Dabei kann ich mich kaum noch an ihn erinnern, dort ist vielleicht ein fröhliches Lächeln, die kurzen Haare und eine Brille. \\\\
Das Kino verblasste mit den Jahren, genauso die vielen langen Gespräche. Worte, Sätze, Gesten und die Berührungen versanken mit der Zeit wie Vogelfedern im Schlamm, in der Tiefe meines Herzens. 
Es sollte ein besonderer Tag für die Kinder und mich werden, extra nur für uns. Versprochen.\\\\
Kurz musste er dann schlagartig etwas holen. \\\\
Es waren die weißen Federbetten, die seine Frau im Geschäft für Weihnachten gekauft hatte und die er für sie abholen sollte.\\\\
Nur, wenn ich mit dem Fingern im Schlamm wühle und die zarten Federn eine, nach der anderen hervorziehe, dann sehe ich sie wieder die fröhlichen Gesichter meiner Kinder durch den Rückspiegel im Auto, das sich wiederholende Spiel vom „Ich sehe was, was du nicht siehst“. Gleichzeitig erscheint er mir in einem Schatten, gesichtlos mit der Unruhe die aufsteigt und sich ausbreitet, den Ganghebel und das Gaspedal umschließt, fordernd zur schnellen Rückkehr. Während die Dunkelheit bereits den schwarzen Theatervorhang am Tal zugezogen hat, verschwindet der Fluss, verschwinden die Häuser da draußen, die Menschen, nur das Amaturenbrett leuchtet in bunten Lichtern. Mein Licht hatte in dieser Stunde an Kraft verloren. \\\\
Diese Erkenntnis erfüllte mich mit fast ebenso unerträglicher Trauer wie das Wissen, dass er mich belogen hatte.\\\\
Dies war der Beginn vom Ende meiner Liebesgeschichte.