\chapter{Der Kriminal Tango}
%\vspace{\baselineskip}
\lettrine{B}{ei} den Koflers im Haus Dr. Kofler stand im Wohnzimmer ein Plattenspieler mit Lautstärkerboxen. \\\\
\textit{An diesem warmen Sommertag ging ich barfuß durch den penibel gepflegten Garten meines Freundes Burkhart aus Kinderzeit, vorbei an exotischen Gartensträucher, blühenden englischen Rosenstöcken, durch das kleine weiß gestrichene Gartentor. Ich hasste Sandalen im Sommer, wenn sich die Steinchen zwischen den Zehen festbohrten und Stutzen trug man nur an Sonn-und Feiertagen zum feschen Dirndl oder Rock. Und Kleider verabscheute ich sowieso - Schuhe brauchte man eh nur im Winter. \\\\
Bereits auf dem ersten Blick unterschied sich das Haus der Koflers von den anderen in unserem Quartier in der Neunerstraße Nr. 15. Ein Bungalow mit Flachdach in unserer bergigen Gegend in leuchtendem Weiß, mit reichlich enormen Fenstern und wenigen Mauern. Auf der großen Terrasse stand eine Hollywoodschaukel, die von einem Rasen in gründlichem Kurzschnitt eingefasst war. Im südlichen Eck wippten die Zweige zweier zarten Birken. Ich grüßte freundlich zu den Blättern, die bald im Herbst welk von den Bäumen auf den schönen Grasteppich fallen würden.\\\\
Im Wohnzimmer stand ein eindrucksvolles Sofa mit weichen Polstern. Ein beträchtlicher Beistelltisch fußte auf einem flauschigen weißen Teppich mit langen dicken Haaren. \\\\
Marianne die Haushälterin schaute mit ihrem prüfenden Scharfblick auf meine schmutzige Füße.\\\\ 
Später hockte ich brav, die Füße eine handbreit außerhalb des weißen Teppichs auf der fremden Couch im Wohnzimmer. Den Rücken durchgedrückt steif, bewegungslos. Der Umstand, dass ich überhaupt reindurfte tröstete meinen krampfenden Magen und die einschlafenden Beine. \\\\
Burkhart hatte eine neue große Schallplatte bekommen. „Der Zauberer von Oz“ ein Hörspiel. Sie steckte in einer bunten Hülle, die schwarze Platte leuchtete frisch. \\\\
Die tiefe Stimme des Erzählers aus den Boxen füllte den Raum mit der Geschichte von Dorothy aus Kansas, wie sie auf die Reise ging mit dem Blechmann dem das Herz fehlt, der Vogelscheuche ohne Verstand und dem feigen Löwen. \\\\
Irgendwo zwischen den Kapiteln musste ich gehen.
Am gleichen Abend holte meine Mutter unseren Plattenspieler hervor, ein Holzding das zugleich als Radio diente und legte eine der kleinen Platten oben auf. Eine orange Scheibe drehte sich kratzend und knarrend auf dem Teller.\\}
\begin{center}
	\textbf{Der Kriminal-Tango}\\\\
„Kriminal-Tango in der Taverne\\
Dunkle Gestalten und rote Laterne\\ 
Und sie tanzen einen Tango \\
Glühende Blicke, steigende Spannung \\
Und in die Spannung, da fällt ein Schuß“\\\\
\end{center}

\textit{Und bei dem Wort Schuss nahmen meine Mutter und ich uns bei den Händen und tanzten zusammen.\\\\
Ich mit den verdreckten Fußsohlen auf dem Küchenboden.\\\\
}
Glück kann so einfach sein