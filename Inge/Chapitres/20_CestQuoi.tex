\chapter{Tante Frieda}
%\vspace{\baselineskip}
\lettrine{E}{inmal} im Jahr fuhr die Mutter mit den vier Kindern, drei Buben und einem Mädchen zu ihrer Schwester - für sie Tante Frieda - nach Meran. \\\\
\textit{Es war einer dieser tiefblauen wolkenlosen Tage, wo die klare Luft endlos schien und die Schwalben über die Dächer fegten mit den Stimmen des nahenden Sommers.\\\\
Der Zug rollte gleich einer müden Stahlschlange durch die engen Kurven. Wir hatten es uns im Abteil des Treno ferrovia bequem gemacht. Sechs Plätze, ein Schiebefenster und eine Schiebetür. Die Vorfreude auf das obligatorisch gegrillte Hühnchen mit Bratkartoffeln machte uns ungeduldig. Wir liefen auf den Gang, rückten die Vorhänge, rüttelten die Schiebetür, zappelten auf den Sitzen, bohrten in den Nasen und traten uns gegen das Schienbein, bis zur allerletzten Ermahnung der Mutter. \\\\
Zwischen den Geleisen, der Autobahn und der Landstraße schlängelte sich der Fluss zwischen den wenigen aufsteigenden hellgrünen Wiesen und den dunkelgrün der Wälder durch das enge Eisacktal, dass umrahmt war von den Dörfern auf beiden Seiten, hoch auf den Hängen, mit den hervorragenden Kirchtürmen, den wuchtigen Bauernhöfen und bewirtschafteten Feldern. Unter den Spitzen der Bergkämme träumten die baumlosen Almen nach der Schneeschmelze vor sich hin. Vorbei am prächtigen Kloster Säben, vorbei an den Häusern mit Giebeldächern und bepflanzten Balkonen. Mehr und mehr grub sich der Zug durch die enge Schlucht vor die Hauptstadt Bozen, dazwischen die bedrohlichen dürren Steilhänge, gräuliche Teppiche aus Mischwald. Dann öffnete sich endlich das Tal, wurde breiter und es zeigten sich in vielen Reihen Weinreben und Obstbäume, in deren Ebenen und Mulden die verschlafenen Dörfer den Tag begrüßten. Am Horizont blitzte von den hohen Bergspitzen in der Ferne, der flammende Schnee.\\\\
\textbf{Wir zählten, eins, zwei, drei, vier...blühende Bäume. Die Welt wurde hier für uns größer.}\\\\
Wie auf glühenden Kohlen, sittsam und brav die Köpfe gesenkt, saßen wir dann in der Stube, horchten den Gesprächen der Erwachsenen, kauten am Hühnchen und rührten stumm in den Kakaotassen, während wir Geschwister an die gelben Löwenzahnköpfchen auf den Wiesen und die Eidechsen zwischen den Mauerritzen dachten, da nickten wir nun freundlich und die Fremdheit sank durch Augen und Ohren und setzte sich auf das Herz und drückte hinunter auf unsere Mägen. \\\\
Später am Abend pressten wir die Nasen an das Abteilfenster und starrten zu den Wagonrädern, zugleich der Zug weg-und anrollte zurück in ihre Stadt.\\\\
\textbf{Wir zählten, eins, zwei, drei, vier ... blühende Bäume. Die Welt wurde hier für uns freier.}
}
