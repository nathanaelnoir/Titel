\chapter{Die hungrigen Schuhe
}

\lettrine{E}{s} beginnt schleichend, wie der Nebel der sich sanft auf den See legt. Zuerst sind es die langen Nächte, wo die Mütter an die Kinderbetten eilen, sie in den Kindergarten begleiten, vom Unterricht in den Schulen und Veranstaltungen abholen, kochen, putzen, dann Hausaufgaben und Freunde überwachen und der Blick in den Spiegel wird bedeutungslos. \\\\
\textit{Der Glockenschlag der Turmuhr der Pfarrkirche schlug verhalten durch die Gassen. Der Tag war schwül gewesen. Touristen hatten sich mühevoll durch die Passagen der Stadt von einer Sehenswürdigkeit zur anderen geschleppt. Die wenigen Einheimischen eilten mit ihren Tüten nach Hause. Jetzt am späten Nachmittag zog ein kühles Lüftchen von Norden her über den Brenner und machte es etwas erträglicher. Ein paar Kinder kreischten, genervte Mütter mit Schwimmtaschen tratschten auf einem der Plätze in der Stadt. Ferien.\\\\
Auch sie waren müde. Der Tag war gut gelaufen. Die Mutter und ihre Tochter ergatterten einen der wenigen freien Plätze in ihrem Lieblingskaffe unter den Lauben. Die Tochter rührte und drückte den Strohhalm im Saftglas, die Mutter beobachtete die frechen Spatzen die sich vorsichtig den Tischen näherten in der Hoffnung etwas Essbares zu erbeuten. Auf einem Stuhl lehnten vollgepackt Einkaufstaschen. Die Zufriedenheit gesellte sich zu ihnen und berauschte ihre Gespräche. \\\\
Die schwarzen Halbschuhe der Mutter waren alt geworden und die Sohle des rechten Schuhs löste sich bereits auf, wie ein hungriges Maul streckte es seine ledrigen Lippen.\\\\
Sie liefen nebeneinander, da hakte sich die Tochter unter und schleppte die Mutter schließlich vor das Schaufenster des Schuhgeschäfts, gestikulierte, überzeugte, drängte, drückte ihre Hand, und schubste sie letztendlich durch die Geschäftstür. \\\\
Die braunen Halbschuhe mit der golden Schnalle und dem gewagten Absatz standen dann später im Flur. Es folgten künftig noch welche in blassem lila, ein paar Stiefeletten und hohe Stiefel.}\\\\

Wärmende Sonnenstrahlen am kühlen Morgen lösten den Nebel über dem See und die glatte Oberfläche spiegelte wieder ihr Antlitz.\\\\

Für meine Tochter Viktoria in inniger Dankbarkeit\\
Und du tust es noch immer – gibst mir Mut