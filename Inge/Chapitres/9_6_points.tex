\chapter[Les 6 points pour l'avenir de la zad]{Les 6 points\\pour l'avenir de la zad} 
%\vspace*{-1.2cm}
\centerline{\textbf{\Large Parce qu'il n' y aura pas d'aéroport}}
\vspace{\baselineskip}
\textit{Ce texte en 6 points a pour but de poser les bases communes nécessaires pour se projeter sur la ZAD une fois
le projet d’aéroport définitivement enterré.
Il a été réfléchi au sein d’une assemblée régulière ayant pour objet de penser à l’avenir des terres une fois le
projet d’aéroport abandonné, l'assemblée qui regroupe des personnes issues des différentes composantes du
mouvement de lutte. Ce texte a été longuement débattu, à plusieurs reprises, dans de multiples composantes et
espaces d’organisation du mouvement.\\}

\lettrine{N}{ous} défendons ce territoire et y vivons ensemble de diverses manières dans un riche brassage.
Nous comptons y vivre encore longtemps et il nous importe de prendre soin de ce bocage, de ses
habitant·e·s, de sa diversité, de sa flore, de sa faune et de tout ce qui s'y partage. 
Une fois le projet d’aéroport abandonné, nous voulons :

\begin{enumerate}
\item Que les habitants·e·s, propriétaires ou locataires faisant l'objet d'une procédure d'expropriation ou d'expulsion puissent rester sur la zone et retrouver leur droits.
\item Que les agriculteurs-ices impacté·e·s, en lutte, ayant refusé de plier face
à AGO-VINCI, puissent continuer de cultiver librement les terres dont il-elles ont l’usage, recouvrir leurs droits et poursuivre leurs activités dans
de bonnes conditions.
\item Que les nouveaux habitant·e·s venu·e·s occuper la ZAD pour prendre
part à la lutte puissent rester sur la zone. Que ce qui s’est construit
depuis 2007 dans le mouvement d’occupation en terme d’expérimentations
agricoles hors cadres, d’habitat auto-construit ou d’habitat léger (cabanes,
caravanes, yourtes, etc.), de formes de vie et de lutte, puissent se maintenir
et se poursuivre.
\item Que les  terres pour redistribuées chaque année par la  chambre d’agriculture pour le compte d'AGO-Vinci sous la forme de baux précaires soient prises en charge par une entité issue du mouvement de lutte qui rassemblera toutes ses composantes. Que ce soit donc le mouvement anti-aéroport et non les institutions habituelles qui déterminent l’usage de ces terres.
\item Que ces terres aillent à de nouvelles installations agricoles et non
agricoles, officielles ou hors cadre, et non à l’agrandissement.
\item Que ces bases deviennent une réalité par notre détermination
collective. Et nous porterons ensemble une attention à résoudre les
éventuels conflits liés à leurs mise en œuvre.
\end{enumerate}

Nous semons et construisons déjà un avenir sans aéroport dans la diversité et la cohésion. C’est
à nous tout·e·s, dès aujourd’hui, de le faire fleurir et de le défendre.